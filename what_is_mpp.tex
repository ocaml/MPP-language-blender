% -*- coding: utf-8; -*-
\documentclass[a4paper]{article}

\usepackage{fullpage}
\usepackage{listings}
\usepackage{xcolor}
\lstset{
  basicstyle=\ttfamily,
  backgroundcolor=\color{black!10},
}

\title{MPP: a meta preprocessor and a language blender}

\author{Philippe Wang}


\def\eg{\emph{e.g.}}

\begin{document}
\maketitle
\begin{center}
  OCaml Labs\\
  The University of Cambridge Computer Laboratory, UK\\
\end{center}
\paragraph{MPP is yet another preprocessor.}
As some other few preprocessors, MPP
is  designed  to  be  generic  and  to be  used  with  any  text-based
programming  language (e.g.,  ASM, C,  Java, ML,  Scheme)  or document
description language  (e.g., HTML, LaTeX, Markdown). MPP  may be given
parameters in  order to restrict or  loosen its usage. It  can also be
extended to provide additional features.

\paragraph{MPP  is  a tool  designed  to  be used  by  both  non-programmers  and
programmers.} First, it means that it  has to be easy to use. For that,
the  syntax was  made very  simple and  there is  very little  need to
understand  what   ``programming''  means.   Second,   it  means  that
programmers should not be frustrated  when using it. For that, MPP can
be easily  extended and be  used to generate  a customized/specialized
preprocessor.

MPP takes a file and processes its contents.  The output is the output
from interpretation of special commands  and verbatim copy the rest of
the  file.  The  usage may  be customized,  by using  special built-in
commands,  command line  options, or  by modifying  the implementation
(which is designed to be easy to modify).

Here follows an example of a LaTeX file intended to be processed by MPP,
where tokens \texttt{((} and \texttt{))} are used to delimit a special 
command:
\begin{lstlisting}
Executing the following command \texttt{mkdir -p a/b/c} 
% (( cmd mkdir -p a/b/c ))
creates a few directories.
Here's the result of running the \texttt{tree} command afterwards:
\begin{verbatim}
(( cmd tree ))
\end{verbatim}
\end{lstlisting}

and the output is then:
\begin{lstlisting}
Executing the following command \texttt{mkdir -p a/b/c} 
% 
creates a few directories.
Here's the result of running the \texttt{tree} command afterwards:  
\begin{verbatim}
.
`-- a
    `-- b
        `-- c

3 directories, 0 files
\end{verbatim}
\end{lstlisting}
This example used the \texttt{cmd} built-in command to execute an external
shell command.

\paragraph{MPP  is a language  blender} as  it allows  any programming
language to  be smoothly used  as a preprocessing language.   Note that
languages  such as  Perl  or Bash  also  allow this  kind of  practice
however  MPP is  designed to  make  it a  lot easier,  safer and  more
relevant.

Let Tyu be a programming  language or a document description language.
Let L be a programming language.   Say that one is writing a .tyu file
and  wants  to  use  the  L programming  language  as  a  preprocessing
language. 

MPP makes this possible if it has been given a description to allow it
to convert the  .tyu file into a program in the  L language that, when
executed, prints back a .tuy  file. Inside the original .tyu file, one
may  declare special  blocks and  use the  L language,  so  that those
blocks are not converted but simply copied verbatim.
This way, all blocks written in the L language share the same execution
environment.

For instance, if TUY=HTML and L=OCaml:

Here follow the contents of file \texttt{f.html} to be processed by MPP and produces
\texttt{tmp.ml}:
\begin{lstlisting}
<table>
{{
let () = 
for i = 1 to 5 do
 Printf.printf "<tr>";
 for j = 1 to 5 do
  Printf.printf "<td>%d</td>" (i*j)
 done;
 Printf.printf "</tr>\n";
done
}}
</table>
\end{lstlisting}

Here follow the contents of \texttt{tmp.ml}:
\begin{lstlisting}
let () = Printf.printf "<table>\n"
let () = 
for i = 1 to 5 do
 Printf.printf "<tr>";
 for j = 1 to 5 do
  Printf.printf "<td>%d</td>" (i*j)
 done;
 Printf.printf "</tr>\n";
done
let () = Printf.printf "</table>\n"
\end{lstlisting}

The resulting file is:
\begin{lstlisting}
<table>
<tr><td>1</td><td>2</td><td>3</td><td>4</td><td>5</td></tr>
<tr><td>2</td><td>4</td><td>6</td><td>8</td><td>10</td></tr>
<tr><td>3</td><td>6</td><td>9</td><td>12</td><td>15</td></tr>
<tr><td>4</td><td>8</td><td>12</td><td>16</td><td>20</td></tr>
<tr><td>5</td><td>10</td><td>15</td><td>20</td><td>25</td></tr>
</table>
\end{lstlisting}

As one  can see, tokens \verb+{{+  and \verb+}}+ allowed  to declare a
special section  in file \texttt{f.html}.  Note that  these two tokens
may be customized by the user of MPP when calling it.

By invoking MPP multiple times with different block delimiters, one
may blend as many languages as he/she may want to.

\paragraph{Notes on MPP}
The commands part and the language  blending part of MPP may happen in
the same MPP execution or separately, this choice is up to the user.

By default, MPP stops if any error occurs (\eg, access to an unbound
variable, or external program exiting with nonzero), but options are
provided to ignore such errors.

\paragraph{About the implementation of MPP}
MPP is implemented in OCaml and its set of features can be extended by
registering additional features that are simple OCaml functions.

MPP may also be used as a library for its simple generic lexing/parsing
engine.

MPP is not taking the approach of embedding something into OCaml. Instead,
it allows to embed the language of your choice into any existing document.

MPP  allows OCaml  to  be used  as  a preprocessing  language for  any
program  or document  text file.  It provides  a generic  mechanism to
allow  the  use  of   other  programming  languages  as  preprocessing
languages.


\paragraph{Conclusion}
MPP is a original and powerful tool for both non-programmers and
programmers.  It is even more powerful for OCaml programmers as they
can easily extend MPP. 

MPP is not intended to replace a tool such as Camlp4 or ppx, which
are meant to extend specifically the OCaml language. MPP extends
any language and is highly customizable.

\end{document}
